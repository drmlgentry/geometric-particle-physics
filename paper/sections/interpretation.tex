\section{Interpretation}

The discovery of a discrete lattice underlying the particle mass spectrum suggests that mass values are not arbitrary outputs of Yukawa couplings, but instead reflect a deeper geometric organization. The appearance of the golden ratio $\varphi$ is particularly striking, as it is a defining feature of icosahedral symmetry and emerges naturally in the representation theory of $A_5$.

In this framework, particle masses are interpreted as logarithmic eigenvalues associated with discrete symmetry sectors. The Higgs mechanism remains responsible for mass generation, but the lattice structure constrains the allowed values once electroweak symmetry breaking has occurred. The mass lattice thus acts as a selection rule rather than a dynamical generator.

Importantly, the construction is parameter-free: no coupling constants or continuous tuning parameters are introduced. The electron mass sets the overall scale, while all relative mass values follow from the lattice geometry alone. This sharply distinguishes the present approach from traditional flavor models that rely on adjustable Yukawa textures.

The absence of degeneracies and the clear geometric clustering by particle type strongly disfavor a coincidental origin. Instead, the data support the hypothesis that a discrete symmetry—possibly a remnant of a higher-energy structure—organizes the mass spectrum.
