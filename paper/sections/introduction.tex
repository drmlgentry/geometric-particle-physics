The Standard Model successfully encodes the gauge interactions of elementary
particles, yet leaves the observed fermion mass spectrum and flavor structure
largely unexplained. Fermion masses span more than twelve orders of magnitude,
exhibiting striking regularities that resist explanation via perturbative
Yukawa couplings alone.

This work advances the hypothesis that fermion masses arise from an underlying
geometric organization in an internal parameter space, rather than from
arbitrary symmetry breaking. Specifically, we propose that the observed mass
spectrum corresponds to discrete points on a logarithmic lattice, generated by
a finite geometric symmetry acting on an internal coordinate space.

We demonstrate that known fermion masses cluster on a structured lattice whose
spacing is controlled by a universal scaling constant. This lattice structure
is independent of gauge charge and generation, suggesting a geometric origin
distinct from Standard Model dynamics. The framework yields falsifiable
predictions for new states and mass relations, offering a concrete pathway for
experimental verification.
\section{Introduction}

The origin of fermion and boson mass hierarchies in the Standard Model remains an open structural problem. 
While the Higgs mechanism successfully explains the generation of mass through electroweak symmetry breaking, 
it does not account for the observed logarithmic spacing of particle masses across many orders of magnitude, 
nor for the apparent regularities within and across flavor sectors.

A large body of work has explored discrete flavor symmetries, including $A_4$, $S_4$, and $A_5$, 
motivated by their natural appearance in lepton mixing and neutrino phenomenology. 
Independently, geometric and number-theoretic approaches have suggested that fermion masses may exhibit 
hidden regularities when expressed in logarithmic or scale-invariant variables.

In this work, we identify a simple and robust empirical structure: when particle masses are expressed 
logarithmically relative to a fixed reference scale, they cluster near integer-valued indices that admit 
a natural embedding into a low-dimensional lattice associated with the alternating group $A_5$. 
The appearance of this structure is independent of model-specific assumptions about Yukawa couplings 
or Higgs-sector dynamics.

The purpose of this paper is deliberately narrow. We do not propose a new mechanism of mass generation, 
nor do we claim a complete theory of flavor. Instead, we demonstrate that the observed particle spectrum 
admits a concise geometric organization, which may serve as a symmetry-based constraint on more 
fundamental theories.

The paper is organized as follows. Section~\ref{sec:a5} reviews the essential mathematical structure 
of the alternating group $A_5$ and its relevance to geometric embeddings. 
Section~\ref{sec:massmodel} defines the logarithmic mass index and the associated lattice construction. 
Section~\ref{sec:results} presents the empirical results and visualizations. 
Section~\ref{sec:interpretation} discusses the possible physical significance and limitations of the observed structure.
