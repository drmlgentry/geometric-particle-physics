The alternating group $A_5$, the group of even permutations of five elements,
is the smallest non-Abelian simple group and plays a central role in discrete
geometric symmetries. It possesses 60 elements and is isomorphic to the group
of orientation-preserving symmetries of the regular icosahedron.

We interpret $A_5$ not as a flavor symmetry imposed by hand, but as a geometric
automorphism group acting on an internal mass-coordinate lattice. The
irreducible representations of $A_5$ naturally organize fermion states into
orbits whose spacing reflects the underlying geometry.

In particular, the appearance of the golden ratio $\varphi = (1+\sqrt{5})/2$
is not incidental: it arises from the eigenvalues of rotation operators acting
on the icosahedral geometry. These eigenvalues determine logarithmic mass
separations, leading to exponential scaling relations observed empirically in
the fermion spectrum.

This interpretation reframes discrete symmetry groups as generators of internal
geometry rather than purely algebraic flavor labels.
