\section{The Alternating Group $A_5$}
\label{sec:a5}

The alternating group $A_5$ is the group of even permutations of five elements and is the smallest 
non-abelian simple group, with order $|A_5| = 60$. 
It is isomorphic to the rotational symmetry group of the regular icosahedron and dodecahedron, 
making it uniquely positioned at the intersection of algebra and geometry.

$A_5$ admits five irreducible representations: 
one trivial singlet, two inequivalent three-dimensional representations, one four-dimensional representation, 
and one five-dimensional representation. 
The presence of multiple low-dimensional irreducible representations distinguishes $A_5$ from smaller 
finite groups and has motivated its repeated appearance in flavor-symmetry constructions.

A notable geometric feature of $A_5$ is its intrinsic association with the golden ratio
\begin{equation}
\varphi = \frac{1 + \sqrt{5}}{2},
\end{equation}
which appears naturally in the eigenvalues and character tables of its three-dimensional representations. 
This golden-ratio structure is not imposed externally but arises from the symmetry itself.

From a geometric perspective, the action of $A_5$ can be represented as rotations in $\mathbb{R}^3$ 
that preserve an icosahedral lattice. 
Such lattices are not periodic in the conventional crystallographic sense but nevertheless possess 
highly constrained discrete structure.

In this work, $A_5$ is not introduced as a dynamical symmetry acting on fields, 
but rather as a mathematical framework capable of organizing discrete data. 
The relevance of $A_5$ therefore lies in its ability to support a finite, low-dimensional lattice 
into which empirical mass indices may be embedded without fine-tuning.
