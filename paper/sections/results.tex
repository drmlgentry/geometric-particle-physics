\section{Results}
\label{sec:results}

Applying the logarithmic mass index to the known Standard Model particle spectrum yields a strikingly 
compact organization. 
Across more than five orders of magnitude in mass, the indices $q_i$ occupy a narrow integer range 
with clear separation between fermionic generations and bosonic states.

Figure~\ref{fig:qmass} shows the relationship between $q$ and $\log_{10}(m/\mathrm{GeV})$. 
The approximately linear alignment confirms that the integer index faithfully captures 
the hierarchical structure of the spectrum while suppressing irrelevant scale information.

\begin{figure}[H]
\centering
\includegraphics[width=0.75\textwidth]{../figures/q_vs_mass.pdf}
\caption{Logarithmic mass index $q$ versus $\log_{10}(m/\mathrm{GeV})$ for Standard Model particles.}
\label{fig:qmass}
\end{figure}

Embedding the same data into the three-dimensional lattice reveals additional structure. 
Figure~\ref{fig:lattice} shows the spatial separation of leptons, quarks, and bosons, 
with minimal overlap between sectors.

\begin{figure}[H]
\centering
\includegraphics[width=0.75\textwidth]{../figures/mass_lattice_3d.pdf}
\caption{Three-dimensional lattice embedding of particle mass indices, colored by sector.}
\label{fig:lattice}
\end{figure}

No particle was assigned a special role in constructing the lattice, and no parameters were tuned 
to force separation. 
The observed clustering therefore constitutes a nontrivial organizational feature of the spectrum.
