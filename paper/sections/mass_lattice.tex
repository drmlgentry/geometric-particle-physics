\section{The Mass Lattice Construction}

We define a logarithmic mass coordinate
\begin{equation}
q_i \equiv \frac{\log(m_i / m_e)}{\log \varphi},
\end{equation}
where $m_e$ is the electron mass and $\varphi = (1+\sqrt{5})/2$ is the golden ratio.
This definition renders $q_i$ dimensionless and invariant under overall mass rescalings.

Empirically, for all known charged fermions, $q_i$ lies remarkably close to integers.
This motivates the interpretation of fermion masses as occupying discrete sites in a
lattice indexed by $q \in \mathbb{Z}$.

\subsection{Embedding into an $A_5$-Structured Lattice}

The alternating group $A_5$, the rotational symmetry group of the icosahedron,
possesses irreducible representations whose character values naturally involve $\varphi$.
We exploit this structure to embed mass indices into a finite-dimensional lattice
associated with quadratic Casimir invariants.

Each particle $i$ is assigned a lattice coordinate
\begin{equation}
\mathbf{L}_i = (a_i, b_i, c_i) \in \mathbb{Z}^3,
\end{equation}
where $(a_i,b_i,c_i)$ correspond to projections onto inequivalent $A_5$ irreducible sectors.

The embedding is constrained such that
\begin{equation}
q_i = \alpha a_i + \beta b_i + \gamma c_i,
\end{equation}
with fixed coefficients $(\alpha,\beta,\gamma)$ determined once and for all.
No particle-specific parameters are introduced.

\subsection{Empirical Consistency}

Known fermions occupy a sparse subset of lattice sites with no accidental degeneracies.
The absence of observed particles at neighboring allowed sites immediately generates
testable predictions for undiscovered states.
