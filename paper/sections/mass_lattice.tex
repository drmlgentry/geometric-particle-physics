We define an internal mass lattice by associating each fermion with a coordinate
vector $(n_1, n_2, n_3) \in \mathbb{Z}^3$, embedded in a logarithmic mass space.
The physical mass $m$ is given by
\begin{equation}
m = m_0 \exp\left( \alpha_1 n_1 + \alpha_2 n_2 + \alpha_3 n_3 \right),
\end{equation}
where $m_0$ is a reference scale and $\alpha_i$ are fundamental geometric
constants.

Empirically, we find that the dominant contribution arises from a single
effective scaling parameter $\alpha \approx \ln \varphi$, yielding
\begin{equation}
m_n = m_0 \varphi^n.
\end{equation}

Known fermion masses align with integer or half-integer values of $n$ within
experimental uncertainty. Deviations correspond to radiative corrections and
electroweak symmetry effects, which enter multiplicatively rather than
structurally.

The lattice interpretation explains both the hierarchical ordering and the
relative stability of mass ratios across generations.
