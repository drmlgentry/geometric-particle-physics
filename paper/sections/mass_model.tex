\section{Logarithmic Mass Index and Lattice Construction}
\label{sec:massmodel}

To expose scale-invariant structure in the particle spectrum, we define a logarithmic mass index
\begin{equation}
q_i = \mathrm{Round}\!\left( \alpha \, \log\!\frac{m_i}{m_0} \right),
\end{equation}
where $m_i$ is the physical mass of particle $i$, $m_0$ is a fixed reference mass, 
and $\alpha$ is a dimensionless normalization constant chosen to map the observed spectrum 
onto a compact integer range.

The specific choice of $m_0$ and $\alpha$ does not affect the qualitative conclusions of this work, 
provided they are fixed globally across all particle sectors. 
In practice, $m_0$ is taken to be the electron mass, and $\alpha$ is chosen such that 
the heaviest known particles correspond to $q = \mathcal{O}(10^2)$.

The integer-valued indices $q_i$ define a discrete ordering of particle masses that is insensitive 
to small experimental uncertainties. 
Crucially, this construction introduces no free parameters per particle; the same mapping applies 
uniformly to leptons, quarks, and electroweak bosons.

To visualize geometric structure, each particle is further assigned coordinates $(a_i,b_i,c_i)$ 
in a three-dimensional lattice. 
These coordinates are constructed as integer-valued functions of $(q_i, \mathrm{sector}, \mathrm{generation})$, 
chosen to preserve adjacency relations between nearby mass indices.

The resulting lattice is not assumed \emph{a priori} to possess any symmetry. 
Its relevance lies entirely in the empirical observation that particles cluster into 
well-separated regions corresponding to known physical sectors.
